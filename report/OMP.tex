\subsection{Open MP}
OpenMP which stands for open multiprocessing is an API (Application program interface) that can be used for creating multi-threaded programs. The interface allows for the explicit creation of threads and utilizes shared memory for the parallelism.

The API is specified for C, C++ and Fortran, it works with most major platforms including Linux and Windows.It also acts to provide a standard for the use of shared memory parallelisation over a range of architectures.

The system works by running a single master thread until the program encounters an area of the code specified as a parallel section. At this point the master thread spawns a new team of threads of which it is the master. Once the thread reaches the end of the parallel section it acts as if this point is a barrier until the rest of the threads join it at which point the system returns to running only the master thread. 

The way that the tasks are split up between the threads depends on the type of parallelisation specified. The simplest is done by unrolling for / DO loops that don’t depend on the outcome of previous loops. This is the parallelisation that was used extensively in splitting up the for loops in the convolution of images for this project. The code can also be broken into sections of which each thread executes one.Inside a parallel section a section of serialized code can also be inserted if required. Parallel sections may be specified inside other parallel sections to create more threads as required.

For this project one of the key differences of OpenMP is that it can not be used on distributed memory parallel systems without use of other systems (such as MPI). This means that the de-blurring would have to run on a single system increasing the minimum time that could be expected to be achieved through the limited parallelisation to the code imposed by the shared memory.