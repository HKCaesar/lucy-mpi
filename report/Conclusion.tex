\section{Conclusion}

Three computer vision solutions were looked at for tracking a marker for use in allowing a kart to follow a lead vehicle. Colour detection was found to be light weight, robust to movement and reasonably accurate in constant lighting (error of less than 7\%). It was however plagued by a large sensitivity to lighting changes that made it unusable as a robust marker location system. The SURF algorithm ran slowly and its inability to handle distances over 1m and even small amounts of motion blur made it inappropriate for the application of being mounted to a moving vehicle. Chessboard detection was the only system that was robust enough to detect a marker and give its position reliability. It was slightly hindered by it inability to deal with large amounts of motion blur and large distances. The output of these systems were combined with sensors measuring the speed and wheel angle of the kart to produce a system that followed the path the marker had taken rather than just driving at the marker giving an accurate following method that could in the tests conducted effectively follow a marker placed on a leading vehicle. This following method was never tested on the kart however due to the kart’s drive by wire systems not being finished before the end of the project.